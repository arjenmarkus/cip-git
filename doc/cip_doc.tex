\documentclass[onecolumn]{article}
\begin{document}

\title{Calculating in Parallel - a simple tool}

\author{Arjen~Markus}

\maketitle

\section{Introduction}
The \verb+CIP+ tool (Calculating in Parallel) is meant to solve a
common enough problem: use the computer's resources as efficiently as
possible without too much work. More specifically, it is meant for
this situation:
\begin{itemize}
\item
You need to run a bunch of calculations with the same program, say
various water quality scenarios.
\item
Each calculation takes a fairly long while, a couple of hours perhaps.
\item
The computational program does not use the entire computer, that is,
it does not exploit its multiple processors to speed up the
calculation.
\item
You do not want to sit around and wait for the calculations to be run
one by one.
\end{itemize}
You could write a small batch file or script that simply starts the
various calculations and let it run on. But that might clog up the
computer. And if you use the cluster, you see yourself forced to
use one node per calculation.

Well, enter \verb+CIP+. It helps you manage the calculations by
starting them in the background, but not more than there are
processors around. When a calculation is finished, the next is started
until there are no more calculations left. Management of these
calculations is automatic, you just need to follow a few simple
conventions.

This documentation describes what these conventions are and how
\verb+CIP+ does its work. Hopefully it is a useful tool.


\section{Setting up the calculations}
\verb+CIP+ assumes that the files required as input for your
calculations as well as the output files are generally stored in a
single directory. So what it does is:
\begin{itemize}
\item
Scan the subdirectories of the directory in which it was started
\item
If there is a file called \verb+nocomp.cip+ or \verb+started.cip+ in
that subdirectory, it is skipped. Otherwise a calculation is started
in that subdirectory. The first file indicates that this subdirectory
is not meant for a calculation, the second file means a calculation
has already been started (it may actually have finished).
\item
As soon as a calculation is to be started, a file \verb+started.cip+
is created to indicate this is the case. Then the calculation is
started (see below), the output from the program or programs is stored
in a file \verb+output.cip+ and when the calculation is finished, a
file \verb+done.cip+ is created.
\end{itemize}

Since this procedure is done in parallel "N" times, your program is
started "N" times, each in its own subdirectory and each assigned to
its own processor (\verb+CIP+ does assume that the operating system is
smart enough to do this without any outside help).

The various files are meant to help avoid racing conditions -- you do
not want to start the same calculation twice, certainly not at the
same time -- and to monitor the status of the calculation.

Here is an example of what a directory with several subdirectories
might look like. The calculation is done via a batch file (or shell
script) \verb+runprogram.bat+, which calls \verb+program1+ and
\verb+program2+. The input consists of a file \verb+database+ in a
subdirectory on its own and a specific file \verb+myinput.inp+ in each
of the calculation subdirectories. The directory and its
subdirectories would look something like this:

\begin{verbatim}
scenarios/
    fixed-data/
        database               (Database required by the programs)
        nocomp.cip             (Prevent computation here)
    scen1/
        myinput.inp            (Specific input for scenario "scen1")
    scen2/
        myinput.inp            (Specific input for scenario "scen2")
    scen3/
        myinput.inp            (Specific input for scenario "scen2")

    runprogram.bat             (Batch file to run the programs)
    program1.exe               (Computational program)
    program2.exe               (Postprocessing, creates reports)
\end{verbatim}

The \verb+runprogram.bat+ batch file looks like this:
\begin{verbatim}
program1 %1
if exists output goto postprocessing
goto end
:postprocessing
program2 %1
:end
\end{verbatim}

The batch file takes as argument the name of the input file --
\verb+myinput.inp+ in this case. This name however is fixed, as
\verb+CIP+ has no way to specify such arguments per calculation. If
you need that, you can store a copy of the batch file in the
calculation subdirectories.

\emph{Note:} Since the calculations are done in various subdirectories,
some care must be taken that the relevant programs, batch files and
scripts can all be found. \verb+CIP+ tries to take care of all this by
extending the \verb+PATH+ environment variable, but it does not know
if arguments like \verb+myinput.inp+ refer to a file or not. You may
need to help it by adding ".." or even the complete directory to such
arguments.


\section{Command-line arguments}
\verb+CIP+ recognises the following arguments:

\noindent \verb+-help+ \\
\indent Present an overview of the commands.

\noindent \verb+-status+ \\
\indent Show which calculatons have already been started or done.

\noindent \verb+-clear+ \\
\indent Clear up the working directories so that you can restart all
the calculations. This command removes the files \verb+started.cip+,
\verb+output.cip+ and \verb+done.cip+, but not the file
\verb+nocomp.cip+ in each of the directories.

\noindent \verb+-procs <N>+ \\
\indent Specify the number of simultaneous calculations (one per
processor). If not, \verb+CIP+ will determine the likely number of
useable processors with a simple heuristic method. This option allows
you to control that number.

\noindent \verb+-path <dir>+ \\
\indent Specify a directory to be added to the PATH environment
variable. You can specify one directory per \verb+-path+ argument.
Note: The working directory is prepended after processing
all the arguments.

\noindent \verb+-run+ \\
\indent This argument is used internally -- see section
\ref{technicalDetails}. You should not use it yourself.

\noindent \verb+--+ \\
\indent Indicates the end of the options. Anything after that is
considered part of the command that is to be run. This is especially
useful if your program requires arguments that might be mistaken for
\verb+CIP+ options.

\noindent \emph{anything else} \\
\indent All other arguments are assumed to be part of the command
that is to be run.

In the above example, running \verb+CIP+ can be done by:

\begin{verbatim}
cip -clean runprogram.bat myinput.inp
\end{verbatim}

\noindent or if you want to redo the calculations (after a change in
the program perhaps or a new database):

\begin{verbatim}
cip -clear runprogram.bat myinput.inp
\end{verbatim}

With this command an individual calculation would be more or less
equivalent to the following series of commands:

\begin{verbatim}
cd scen1
echo Started >started.cip
runprogram.bat myinput.inp >output.cip
echo Done >done.cip
\end{verbatim}


\section{Some technical details}
\label{technicalDetails}
To better understand what \verb+CIP+ is doing, here are a few details:
\begin{itemize}
\item
The number of processors is determined on Windows via the
\verb+NUMBER_OF_PROCESSORS+ environment variable. This is a simple,
but not entirely reliable method. The alternatives I have been able to
find give unexpected and completely useless results.

On Linux it counts the number of lines that contain "processor :" in
the pseudo-file \verb+/proc/cpuinfo+.

Whatever the method, hyperthreading will report a single hardware
processor as two or more processors, whereas the effective number of
processors might be less than that, as hyperthreading only makes a
single processor appear to be several processors. There is no way of
examining this that I know of. You might want to specify a lower
number of processors to use to prevent slowdown of your computer.

\item
\verb+CIP+ clones itself "N" times. These clones use the special
argument, \verb+-run+, as an indication that they should actually run
the program. This way the operating system does the actual parallel
processing and each clone can simply wait for the
computational program to finish before starting the next.

\item
The files \verb+started.cip+ are created as "atomically" as possible,
so that race conditions are avoided as much as possible. It is
thus quite unlikely that two clones are able to start a calculation in
the same directory.

\item
If two or more computers have access to the same working directory,
for instance because it resides on a network disk, then you can
actually start \verb+CIP+ on each of these computers and gain even
more efficiency.

\item
\verb+CIP+ comes in a Windows version and a Linux version.
\end{itemize}

\end{document}
